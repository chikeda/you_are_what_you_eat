\documentclass[]{article}
\usepackage{lmodern}
\usepackage{amssymb,amsmath}
\usepackage{ifxetex,ifluatex}
\usepackage{fixltx2e} % provides \textsubscript
\ifnum 0\ifxetex 1\fi\ifluatex 1\fi=0 % if pdftex
  \usepackage[T1]{fontenc}
  \usepackage[utf8]{inputenc}
\else % if luatex or xelatex
  \ifxetex
    \usepackage{mathspec}
  \else
    \usepackage{fontspec}
  \fi
  \defaultfontfeatures{Ligatures=TeX,Scale=MatchLowercase}
\fi
% use upquote if available, for straight quotes in verbatim environments
\IfFileExists{upquote.sty}{\usepackage{upquote}}{}
% use microtype if available
\IfFileExists{microtype.sty}{%
\usepackage{microtype}
\UseMicrotypeSet[protrusion]{basicmath} % disable protrusion for tt fonts
}{}
\usepackage[margin=1in]{geometry}
\usepackage{hyperref}
\hypersetup{unicode=true,
            pdftitle={You Are What You Eat: An analysis of the efficacy of folic acid food fortification initiatives on the prevalence of neural tube defects in newborns},
            pdfauthor={Chad Ikeda},
            pdfborder={0 0 0},
            breaklinks=true}
\urlstyle{same}  % don't use monospace font for urls
\usepackage{color}
\usepackage{fancyvrb}
\newcommand{\VerbBar}{|}
\newcommand{\VERB}{\Verb[commandchars=\\\{\}]}
\DefineVerbatimEnvironment{Highlighting}{Verbatim}{commandchars=\\\{\}}
% Add ',fontsize=\small' for more characters per line
\usepackage{framed}
\definecolor{shadecolor}{RGB}{248,248,248}
\newenvironment{Shaded}{\begin{snugshade}}{\end{snugshade}}
\newcommand{\KeywordTok}[1]{\textcolor[rgb]{0.13,0.29,0.53}{\textbf{#1}}}
\newcommand{\DataTypeTok}[1]{\textcolor[rgb]{0.13,0.29,0.53}{#1}}
\newcommand{\DecValTok}[1]{\textcolor[rgb]{0.00,0.00,0.81}{#1}}
\newcommand{\BaseNTok}[1]{\textcolor[rgb]{0.00,0.00,0.81}{#1}}
\newcommand{\FloatTok}[1]{\textcolor[rgb]{0.00,0.00,0.81}{#1}}
\newcommand{\ConstantTok}[1]{\textcolor[rgb]{0.00,0.00,0.00}{#1}}
\newcommand{\CharTok}[1]{\textcolor[rgb]{0.31,0.60,0.02}{#1}}
\newcommand{\SpecialCharTok}[1]{\textcolor[rgb]{0.00,0.00,0.00}{#1}}
\newcommand{\StringTok}[1]{\textcolor[rgb]{0.31,0.60,0.02}{#1}}
\newcommand{\VerbatimStringTok}[1]{\textcolor[rgb]{0.31,0.60,0.02}{#1}}
\newcommand{\SpecialStringTok}[1]{\textcolor[rgb]{0.31,0.60,0.02}{#1}}
\newcommand{\ImportTok}[1]{#1}
\newcommand{\CommentTok}[1]{\textcolor[rgb]{0.56,0.35,0.01}{\textit{#1}}}
\newcommand{\DocumentationTok}[1]{\textcolor[rgb]{0.56,0.35,0.01}{\textbf{\textit{#1}}}}
\newcommand{\AnnotationTok}[1]{\textcolor[rgb]{0.56,0.35,0.01}{\textbf{\textit{#1}}}}
\newcommand{\CommentVarTok}[1]{\textcolor[rgb]{0.56,0.35,0.01}{\textbf{\textit{#1}}}}
\newcommand{\OtherTok}[1]{\textcolor[rgb]{0.56,0.35,0.01}{#1}}
\newcommand{\FunctionTok}[1]{\textcolor[rgb]{0.00,0.00,0.00}{#1}}
\newcommand{\VariableTok}[1]{\textcolor[rgb]{0.00,0.00,0.00}{#1}}
\newcommand{\ControlFlowTok}[1]{\textcolor[rgb]{0.13,0.29,0.53}{\textbf{#1}}}
\newcommand{\OperatorTok}[1]{\textcolor[rgb]{0.81,0.36,0.00}{\textbf{#1}}}
\newcommand{\BuiltInTok}[1]{#1}
\newcommand{\ExtensionTok}[1]{#1}
\newcommand{\PreprocessorTok}[1]{\textcolor[rgb]{0.56,0.35,0.01}{\textit{#1}}}
\newcommand{\AttributeTok}[1]{\textcolor[rgb]{0.77,0.63,0.00}{#1}}
\newcommand{\RegionMarkerTok}[1]{#1}
\newcommand{\InformationTok}[1]{\textcolor[rgb]{0.56,0.35,0.01}{\textbf{\textit{#1}}}}
\newcommand{\WarningTok}[1]{\textcolor[rgb]{0.56,0.35,0.01}{\textbf{\textit{#1}}}}
\newcommand{\AlertTok}[1]{\textcolor[rgb]{0.94,0.16,0.16}{#1}}
\newcommand{\ErrorTok}[1]{\textcolor[rgb]{0.64,0.00,0.00}{\textbf{#1}}}
\newcommand{\NormalTok}[1]{#1}
\usepackage{graphicx,grffile}
\makeatletter
\def\maxwidth{\ifdim\Gin@nat@width>\linewidth\linewidth\else\Gin@nat@width\fi}
\def\maxheight{\ifdim\Gin@nat@height>\textheight\textheight\else\Gin@nat@height\fi}
\makeatother
% Scale images if necessary, so that they will not overflow the page
% margins by default, and it is still possible to overwrite the defaults
% using explicit options in \includegraphics[width, height, ...]{}
\setkeys{Gin}{width=\maxwidth,height=\maxheight,keepaspectratio}
\IfFileExists{parskip.sty}{%
\usepackage{parskip}
}{% else
\setlength{\parindent}{0pt}
\setlength{\parskip}{6pt plus 2pt minus 1pt}
}
\setlength{\emergencystretch}{3em}  % prevent overfull lines
\providecommand{\tightlist}{%
  \setlength{\itemsep}{0pt}\setlength{\parskip}{0pt}}
\setcounter{secnumdepth}{0}
% Redefines (sub)paragraphs to behave more like sections
\ifx\paragraph\undefined\else
\let\oldparagraph\paragraph
\renewcommand{\paragraph}[1]{\oldparagraph{#1}\mbox{}}
\fi
\ifx\subparagraph\undefined\else
\let\oldsubparagraph\subparagraph
\renewcommand{\subparagraph}[1]{\oldsubparagraph{#1}\mbox{}}
\fi

%%% Use protect on footnotes to avoid problems with footnotes in titles
\let\rmarkdownfootnote\footnote%
\def\footnote{\protect\rmarkdownfootnote}

%%% Change title format to be more compact
\usepackage{titling}

% Create subtitle command for use in maketitle
\newcommand{\subtitle}[1]{
  \posttitle{
    \begin{center}\large#1\end{center}
    }
}

\setlength{\droptitle}{-2em}
  \title{You Are What You Eat: An analysis of the efficacy of folic acid food
fortification initiatives on the prevalence of neural tube defects in
newborns}
  \pretitle{\vspace{\droptitle}\centering\huge}
  \posttitle{\par}
  \author{Chad Ikeda}
  \preauthor{\centering\large\emph}
  \postauthor{\par}
  \date{}
  \predate{}\postdate{}


\begin{document}
\maketitle

\subsection{Research in context}\label{research-in-context}

Folic acid is an important B vitamin, with particular importance for
pregnant women. This vitamin is implicated in proper brain/spinal
development in newborns. Supplementation of folic acid in pregnant women
has been shown to reduce the rate of neural tube defects {[}1{]}.
Certain countries have enacted policies mandating the fortification of
certain foods with folic acid, while others have not. I aim to quantify
the effect of these policies, by conducting a difference-in-difference
analysis.

\subsubsection{Methods}\label{methods}

At a high-level, the difference-in-difference method molds observational
data into the structure of an experiment/control setup based on
admistrition of treatment. The treatment effect is then calculated by
comparing the differences in the slopes of the observed experimental
group (post-treatment), and the counterfactual slope calculated using
the slope of the control group. The idea being: we cannot truly observe
the counterfactual of the experimental group, but using a control group
that does not experience the treatment is a decent proxy. The
experimental and control groups will still vary from each other, but
this method aims to control for these initial differences.

\subsection{Background}\label{background}

This project draws upon my work at IHME. Recently, we have partnered
with the Food Fortification Initiative (FFI) {[}2{]}, ``an international
partnership working to improve health through fortification of
industrially milled grain products''. From their data, I have generated
a dataset, indexed by country and year, that denotes whether food
vehicles are mandated to be fortified with folic acid. This dataset
serves as my dummy-variable for indicating `treatment', with a 0 for
`no', and a 1 for `yes'. I have selected 15 countries as my initial
sample, varying by region as well as treatment status.

The measure we will be regressing on is the prevalence of neural tube
defects per 10,000. I have pulled these values from IHME's current
`best' models of neural tube defects, for various age/sex groups.

In order to run my analysis, I have created a function that compares 1
experimental and 1 control country, for a given age and sex. A
diff-in-diff regression is run for treatment-status on the prevalence of
neural tube defects. This function returns: 1) a dataframe of
model-results stemming from the regression equation:
\[NTDprev_{y,c} = \beta_0 + \beta_y*Year_y + \beta_c*Country_c + \beta_{diff}*(treatment interaction)\]
for year = y \& country = c, where \(\beta_{diff}\) = the treatment
effect. And 2) a plot showing the observed data and regression lines of:
the control group (blue), the pre-treatment experimental group (red),
and the post-treatment experimental group (green).

\subsubsection{Setup}\label{setup}

\begin{Shaded}
\begin{Highlighting}[]
\KeywordTok{rm}\NormalTok{(}\DataTypeTok{list=}\KeywordTok{ls}\NormalTok{())}
\KeywordTok{library}\NormalTok{(}\StringTok{"tidyverse"}\NormalTok{)}
\KeywordTok{library}\NormalTok{(}\StringTok{"broom"}\NormalTok{)}
\KeywordTok{library}\NormalTok{(}\StringTok{"ggplot2"}\NormalTok{)}

\NormalTok{repodir <-}\StringTok{ "J:/temp/chikeda/2018/180606_Jeff_FinalProj/"}\NormalTok{ ##replace with filepath of repo and src folder}

\KeywordTok{source}\NormalTok{(}\KeywordTok{paste0}\NormalTok{(repodir, }\StringTok{"diffindiff_function.R"}\NormalTok{))}
\end{Highlighting}
\end{Shaded}

\subsection{Original Computational
Environment}\label{original-computational-environment}

The example we will look at, will be comparing two South American
countries, Mexico and Venezuela. Venezuela is the control (blue), never
having implemented a folic acid fortification initiative, whereas Mexico
is the experimental group, having implemented a policy mandating the
fortification of maize in the year 2000. We will be looking at both
sexes combined, in the Early Neonatal age-group (0-6 days) (note: this
function takes in IHME specific location/age/sex ids. When trying to get
the function to take strings as inputs, it was breaking, so I have left
the syntax as ids. I have provided a data-dictionary. If you would like
to explore these data, please feel free to change the arguments in the
diff\_in\_diff function. There are a handful more locations/ages/sexes
to compare not covered in this narrative.)

\paragraph{Mexico vs.~Venezuela, Both Sexes, Early
Neonatal}\label{mexico-vs.venezuela-both-sexes-early-neonatal}

\begin{Shaded}
\begin{Highlighting}[]
\NormalTok{##function arguments: diff_in_diff(control_location, experimental_location, sex_id, age_group_id)}
\NormalTok{##### if you'd like to analyze different combos, see /data/data_dictionary.csv for explanation of ids and change accordingly}

\KeywordTok{diff_in_diff}\NormalTok{(}\DecValTok{133}\NormalTok{,}\DecValTok{130}\NormalTok{,}\DecValTok{3}\NormalTok{,}\DecValTok{2}\NormalTok{) ## function sourced from diffindiff_function.R above}
\end{Highlighting}
\end{Shaded}

\begin{verbatim}
##                     term     estimate    std.error  statistic      p.value
## 1    location_nameMexico 130.23621517 16.252161960   8.013470 1.212190e-10
## 2 location_nameVenezuela 126.54966751 16.325181011   7.751808 3.146421e-10
## 3                year_id  -0.06124525  0.008148244  -7.516373 7.443389e-10
## 4         mean_maize.csv  -2.18639633  0.194263437 -11.254801 1.499704e-15
\end{verbatim}

\includegraphics{Ikeda_Chad_you_are_what_you_eat_files/figure-latex/unnamed-chunk-2-1.pdf}

But perhaps it is too early to see an effect in the age group of a
child's first week of life. We also have prevalences for children in the
Late Neonatal (7-27 days) and Post Neonatal (28-364 days) age groups.
Let's see if we see any differences here.

\paragraph{Mexico vs.~Venezuela, Both Sexes, Late
Neonatal}\label{mexico-vs.venezuela-both-sexes-late-neonatal}

\begin{Shaded}
\begin{Highlighting}[]
\KeywordTok{diff_in_diff}\NormalTok{(}\DecValTok{133}\NormalTok{,}\DecValTok{130}\NormalTok{,}\DecValTok{3}\NormalTok{,}\DecValTok{3}\NormalTok{) ## function sourced from diffindiff_function.R above}
\end{Highlighting}
\end{Shaded}

\begin{verbatim}
##                     term    estimate    std.error  statistic      p.value
## 1    location_nameMexico 93.69623227 12.200736428   7.679555 4.097038e-10
## 2 location_nameVenezuela 90.52434811 12.255552901   7.386394 1.198422e-09
## 3                year_id -0.04363296  0.006117006  -7.133058 3.035920e-09
## 4         mean_maize.csv -1.66596494  0.145836412 -11.423518 8.609548e-16
\end{verbatim}

\includegraphics{Ikeda_Chad_you_are_what_you_eat_files/figure-latex/unnamed-chunk-3-1.pdf}

\paragraph{Mexico vs.~Venezuela, Both Sexes, Post
Neonatal}\label{mexico-vs.venezuela-both-sexes-post-neonatal}

\begin{Shaded}
\begin{Highlighting}[]
\KeywordTok{diff_in_diff}\NormalTok{(}\DecValTok{133}\NormalTok{,}\DecValTok{130}\NormalTok{,}\DecValTok{3}\NormalTok{,}\DecValTok{4}\NormalTok{) ## function sourced from diffindiff_function.R above}
\end{Highlighting}
\end{Shaded}

\begin{verbatim}
##                     term    estimate   std.error  statistic      p.value
## 1    location_nameMexico 50.51177807 5.985765094   8.438650 2.596722e-11
## 2 location_nameVenezuela 48.26814708 6.012658432   8.027755 1.150813e-10
## 3                year_id -0.02301638 0.003001045  -7.669454 4.251163e-10
## 4         mean_maize.csv -0.83332263 0.071548346 -11.646987 4.151142e-16
\end{verbatim}

\includegraphics{Ikeda_Chad_you_are_what_you_eat_files/figure-latex/unnamed-chunk-4-1.pdf}

\subsubsection{Analysis}\label{analysis}

The coefficients on `mean\_maize.csv' are what we are looking for. I am
still a bit fuzzy on the interpretation of these coefficients, but do
see a trend where they are becoming more positive. I would interpret the
coefficient of `mean\_maize.csv': ``as the treatment variable moves from
0-\textgreater{}1, we see a decrease in the prevalence of neural tube
defects''. As these become more positive, I interpret that as: ``the
effect of folic acid fortification initiatives are most effective the
younger the child is''.

This would make intuitive sense, as we could be more likely to see
differences in prevalence at younger age groups, as children with these
neural tube defects are less likely to live long past birth. We can
likely capture the best `true' prevalence of these defects at the
youngest age group.

\subsection{Limitations \& Future
Directions}\label{limitations-future-directions}

There are MANY ways I'd like to improve this analysis, but believe this
function to be a good starting point. The biggest limitation is my own
understanding of the nuances of this method. I was hoping to do a
multiple diff-in-diff regression, with covariates added in to control
for country-effects and time-effects. However, I was not able to
understand how to get that to work in R.

This analysis only shows a two country comparison, but to get a better
sense of a potential treatment effect, enlarging the country-sample is
the goal. From my understanding, it is necessary to create
dummy-variables for treated/non countries that only turns on when that
country is being analyzed. I believe the same goes for the time-effects
dummy on year\_ids. Ideally, when comparing all 195 countries \&
territories that IHME models for, we would be able to see a better
picture of our `control \& experimental groups'.

The data on food fortification initiatives has data on three food
vehicles: maize, wheat and rice. In this computational narrative, I
focused only on maize. According to the FFI's dataset, maize is the most
common food vehicle to have legislation passed for its mandatory
fortification. It would be interesting to run analyses using all three
of these food vehicles.

Even without these personal decisions that narrowed this analysis, the
next largest limitation would be the data quality itself. The FFI has
admitted their datasets are not complete, and that the data they do have
is `questionable' in places. Our IHME estimates have been in particular
flux this year, and congenital birth defects are notoriously an elusive
model to work with. We have been undergoing changes in the way we model
`birth prevalence', and have revamped the way we use inpatient hospital
data to create our observations. Inpatient hospital data is particularly
tricky to rely on, as it is biased toward hospital accesibility and
feasiblity. We can not really know how many children die during birth in
rural areas, where they cannot be taken to a hospital in time for aid.

Finally, I recognize there are many more controls and covariates that
this model could use to try and discern a clearer picture with more
causal strength. I imagine things such as GDP, access to adequate food
\& water, government functionality or hospital density could all be
intersting things to control for. When working with human health data in
the vein of `causality' there are so many things to consider.
Endogeneity is particularly tough for me to parse out when thinking of
these controls, as well as whether they are pre or post treatment
factors.

\subsubsection{Discussion}\label{discussion}

All in all, this course and final project has exposed me to entirely
novel frameworks of thinking, and has given context to some of the
grayer aspects of my job at IHME, and our institution as a whole.
Through going through the motions of this project, I encountered many
questions that caused me to rethink my analysis. Wondering whether a
government might even enforce a law such as this, for reasons ranging
from low-budgets to aversion to scientific suggestion. Even if a
government decides to enforce a law, whether the manufacturers of said
food vehicle will even agree to change the way they have always done
business, and retrain a factory to incorporate this fortification. Even
if the corporations agree to fortify said food vehicle, wondering
whether grocery stores may stock this product, if it is at a higher cost
than the typical non-fortified option. Even if a grocery store does
agree to stock the fortified food vehicle option, whether the consumer
will pay the potentially higher cost to purchase the fortified option if
it is in competion with other more-affordable options. Even if the
consumer is willing to utilize the fortified food vehicle, will there be
a grocery store within feasible distance for them to procure the
product? EVEN if this consumer finally gets a hold of this fortified
food vehicle, could there be other health-related circumstances that
prevent their body from taking in nutrients at all (such as diarrhea,
poor nutrition of the mother when she was a child, previous pregnancies,
etc.)???

At each step in that chain of questions, I could potentially see a way
to generate a variable that may high-level attempt to capture that
variation. But for each of these variables, similar questions of data
quality would come into play, and it would take much work to collect and
synthesize these data. I see IHME's role differently now, as one of
trying to reveal some of these high-level overarching covariates by
pulling in data from all the sources we can. We try to standardize these
measures to each other, so that one can potentially think up a model
that does its best to tame the chaos in observation. I now see just how
complicated and nuanced this endeavour is, once you know the details
behind the math. Recently, I do worry about our reporting. It's
impossible to know how close (if at all) a given model is to the
`truth', but by espousing estimates and claiming them to be `the most
true', do we spread false information knowingly\ldots{}.? Based on how
much data quality is lost at each step, and the fact that our `models'
take in so many covariates, I worry that our uncertainty propagates
uncertainty, and by the time you reach a final output, the message has
been so bastardized that you might as well be saying nothing at all. I
don't know realistically how you fix this\ldots{}

I think the best thing (and only thing at this juncture) to do, is aim
for transparency. We need to present our research not as just `here are
my results, look how pretty they are', but rather as `this was the
course of my analysis, these are my attempts at controls, and this is
stepwise how they changed the data'. I have shared this project with my
team's lead and our congenital defects researcher/modeler. They were
excited about my analysis, and we hope to refine it for potential use in
a publication going forward. I am now much more aware of the
responsibilty on the researcher for using good methods. You will always
be the most invested and most familiar party with your specific
research. With that comes the immense responsibility to swallow the
tough work of pondering through methods and research design, to digest
your analysis properly for an audience external to yourself. Bad
research is fodder for exploitative policy; we as researchers are
already at such a disadvantage toward legislation enaction and
behavioural change, we need to do the best we can to communicate
properly.

\paragraph{References}\label{references}

{[}1{]} Czeizel, AE. Prevention of the first occurrence of neural-tube
defects by periconceptional vitamin supplementation.
\url{https://www.ncbi.nlm.nih.gov/pubmed/1307234}

{[}2{]} Food Fortification Initiative.
\url{http://www.ffinetwork.org/about/index.html}

Fortification policy data provided to IHME by Becky Tsang of FFI.
Dataset generated by Chad Ikeda. Neural Tube Defects prevalence data
pulled from inpatient hospital claims data, modeled novelly by Lauren
Wilner, IHME.


\end{document}
